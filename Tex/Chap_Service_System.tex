\chapter{基于容器化多终端服务系统架构设计 }\label{chap:service_system}

\section{引言}
% 这一章主要写的就是大背景,边缘设备的空余资源?
% 引言部分要说明目前存着哪些问题
% 需要对空余资源进行管理和利用
% 为何使用去中心化的自组织网络
% 结合微服务特点

本章的内容结构组织如下:第\ref{sec:service_system_related_work}节介绍了一些相关技术的研究工作,包括自治系统的组网技术,微服务架构;第\ref{sec:service_system_design}节介绍了多终端协同服务系统的架构设计,包括引入微服务架构、终端在系统中的身份、系统模块设计等内容;第\ref{sec:service_system_decentralized_network}节提出了多终端协同服务系统的构建方法,包括自治系统的网络选择、自治系统的组网方式、自治系统的管理节点选择等内容;第\ref{sec:service_system_summary}节总结了本章内容。
\section{相关工作}\label{sec:service_system_related_work}
\subsection{自治系统组网技术}



\subsection{微服务架构}


\section{终端服务系统架构设计}\label{sec:service_system_design}
\subsection{微服务架构}
说明微服务架构更适合终端服务系统
\subsection{系统模块设计}
整体结构、模块、架构设计图

什么时候进行迁移、对终端资源的利用

\subsection{终端在系统中的角色}

基于容器化多终端协同服务系统中的终端有三个身份,每个终端需要完成终端本身为用户提供的工作,即“本地服务”,当终端本身资源不足不能够很好的完成用户请求的任务,则应该通过系统中的调度中心向其他空闲节点请求提供相应资源来进行协同服务,而当终端本身资源有剩余的时候,该终端又可以通过调度中心将本身的资源以容器虚拟化的形式向系统中的其他节点提供出去。

在这个系统中,用户、服务、终端、容器这几个概念之间的关系需要解释一下,如图2所示。用户是整个服务过程的发起者,能够通过自己身边的任何一个设备访问该设备提供的服务,用户对服务的每一次访问都是一次请求任务。服务代表终端能够为用户提供某种服务的能力,是一个比较虚的概念,具体包含两种能力:提供该服务应用的虚拟化容器(或虚拟化镜像)和能够支持该容器快速运行的相应资源。新的服务上线需要向系统注册服务信息,提供服务镜像,上报服务运行所需要的相应资源,并暴露相应服务端口。用户访问服务的过程,实质是用户通过服务对外暴露的端口访问部署在终端上的容器,并由终端向用户提供相应的资源。


\section{多终端协同服务系统的构建方法}\label{sec:service_system_decentralized_network}
\subsection{自治系统的网络选择}
\subsection{去中心化的自组织网络构建}

自组织网络的应用遍布于军事、数据通信和突发事件处理等领域,其体系结构是指包含协议和拓扑的网络整体设计,拓扑是自组织网络研究中的典型问题。动态条件下建立满足应用的自组织网络是网络体系结构研究的最终目标。自组织网络与现在大部分网络不一样,由于现阶段网络是一种基于客户端-服务器模式下的网络,通过客户端发送请求,利用服务端接收反馈需求,其中各种网络设备在网络中具有特定角色,而自组织网络中的各种设备是对等的。在信息交互时,自组织网络既可以作为客户端,也可做服务端。因为预先建立的基础骨干网设施还不够完善,所以对于终端系统来说,应该使用去中心化的系统架构。

\subsection{自治系统的管理节点选择}
对于整体而言,利用各个节点间的连接情况来构建一个Connectivity-based Decentralized Node Clustering(CDC)。首先选择若干初始种子节点,初始种子节点的选择算法可以进一步进行研究。每个种子向周围的邻居节点发送消息,消息中会携带ID、种子节点ID、权重、TTL、发送节点相关信息等信息,邻居节点在收到消息后会对该消息做一定处理,并加入节点自身相关的信息,再发送到它的邻居节点。节点消息不断传递下去,直到传递到某个已经确定属于其他集群的节点,或者权重信息消耗尽,或者TTL减小到0,则认为消息传递结束,消息传递过程中经过的节点形成一个集群,集群中的节点互相交换相关信息。在集群内部,可以通过考虑上线时间、稳定程度、负载情况、计算能力、邻居数量、网络状态等信息,选择一个超级节点作为集群的任务调度节点。而当有新的节点上线的时候,不需要重新进行集群的划分,而是应该让该节点向周围邻居节点广播消息,根据网络相关程度、上线时间、地理位置、特殊Tag等方法选择周围邻居节点中的一个加入其所在集群。这样就形成了一个整体去中心化、终端节点自治的一个多终端协同服务系统的底层结构,如图1所示是一个简化的模型。

\section{本章小结}\label{sec:service_system_summary}