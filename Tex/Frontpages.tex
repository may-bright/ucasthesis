%---------------------------------------------------------------------------%
%->> 封面信息及生成
%---------------------------------------------------------------------------%
%-
%-> 中文封面信息
%-
%  %这里加一个全角的空格,可以解决中文复制乱码问题,但是开头会增加两页空白最后可以去掉
\confidential{}% 密级:只有涉密论文才填写
\schoollogo{scale=0.095}{ucas_logo}% 校徽
\title{(评阅版)\\基于容器化多终端协同服务技术研究}% 论文中文题目
\author{赵然}% 论文作者
\advisor{倪\quad 宏~研究员~中国科学院声学研究所}% 指导教师:姓名 专业技术职务 工作单位
\advisors{郭志川~研究员~中国科学院声学研究所}% 指导老师附加信息 或 第二指导老师信息
\degree{博士}% 学位:学士、硕士、博士
\degreetype{工学}% 学位类别:理学、工学、工程、医学等
\major{信号与信息处理}% 二级学科专业名称
\institute{中国科学院声学研究所}% 院系名称
\date{2019~年~6~月}% 毕业日期:夏季为6月、冬季为12月
%-
%-> 英文封面信息
%-
\TITLE{Research on Containerized Multi-terminal \\ Collaborative Service Technology}% 论文英文题目
\AUTHOR{Zhao Ran}% 论文作者
\ADVISOR{Supervisor: Professor Ni Hong \\  \hspace{3.9cm} Professor Guo Zhichuan}% 指导教师
% \ADVISOR{Supervisor: Professor Guo Zhichuan}% 指导教师
\DEGREE{Doctor}% 学位:Bachelor, Master, Doctor。封面格式将根据英文学位名称自动切换,请确保拼写准确无误
\DEGREETYPE{Philosophy}% 学位类别:Philosophy, Natural Science, Engineering, Economics, Agriculture 等
\THESISTYPE{dissertation}% 论文类型:thesis, dissertation
\MAJOR{Signal and Information Processing}% 二级学科专业名称
\INSTITUTE{Institute of Acoustics, Chinese Academy of Sciences}% 院系名称
\DATE{June, 2019}% 毕业日期:夏季为June、冬季为December
%-
%-> 生成封面
%-
\maketitle% 生成中文封面
\MAKETITLE% 生成英文封面
%-
%-> 作者声明
%-
\makedeclaration% 生成声明页
%-
%-> 中文摘要
%-
\chapter*{摘\quad 要}\chaptermark{摘\quad 要}% 摘要标题
\setcounter{page}{1}% 开始页码
\pagenumbering{Roman}% 页码符号

随着边缘计算技术的快速发展,位于网络边缘的用户终端设备在人们的数字化生活中正扮演着越来越重要的角色。
用户终端需要承担越来越多的计算任务,这也对终端设备的计算能力提出了越来越高的要求。
% 而随着芯片技术的迅猛发展,终端设备的计算能力也越来越高。
常用的云计算解决方案在用户终端服务的场景中会带来实时性差、部署成本高、带宽拥挤等问题,利用靠近用户的终端设备上空闲的计算资源协同为用户提供就近计算服务成为一种可能。
% 随着计算机技术的快速发展以及边缘计算技术的逐渐成熟,位于网络边缘的用户终端设备在人们的数字化生活中正扮演着越来越重要的角色。
% 用户终端需要承担越来越多的计算任务,这也对终端设备的计算能力提出了越来越高的要求。
% 常用的云计算解决方案可能会带来实时性差、部署成本高、带宽拥挤等问题,利用靠近用户的终端设备上空闲的计算资源协同为用户提供就近的计算服务成为一种可能的解决方案。

由于多终端的资源分布分散、资源异构性强,存在资源利用难度大及利用率低、服务响应时间长、用户体验较差等问题。为了提高终端资源利用率,提升用户体验,本文开展了基于容器化多终端协同服务技术研究。
% 然而,利用多个终端的资源协同为用户提供服务会面临资源分布较为分散、资源异构性强、资源利用难度大、资源利用率低、费用成本较高、服务响应时间长、用户体验较差等问题。
% 为了合理利用多终端为用户提供服务,提高终端资源利用率,提高用户体验,本文开展了基于容器化多终端协同服务技术研究。
本文首先引入容器技术来解决终端资源的异构和管理问题,设计了基于容器化的多终端协同服务系统。进而,本文提出基于容器的多终端透明计算迁移技术,减少服务响应时间,提高用户体验。然后,本文进行多终端协同服务任务调度算法优化工作,合理分配终端资源,降低终端服务费用,减少服务响应时间。最后,本文研究基于预测的终端弹性服务技术,提高了终端资源利用率。
% 如何将边缘终端设备的空闲资源管理起来并利用多终端设备协同提供服务以达到提高终端资源利用效率、提高终端用户体验的目的,成为本文需要重点解决的问题。
% 为此本文开展基于容器化多终端协同服务技术研究。本文首先介绍了云计算技术、容器虚拟化技术、计算迁移技术、任务调度算法等相关技术和算法的研究现状,分析其目前还存在的问题。本文结合容器虚拟化技术和多终端协同服务技术,设计了基于容器化的多终端协同服务系统,并主要研究系统中的基于容器的多终端透明计算迁移技术、多终端协同服务任务调度算法优化以及基于预测的终端弹性服务技术。

本文主要研究内容与成果如下: 

(1)基于容器化多终端服务系统架构设计

设计了基于容器化多终端协同服务系统,及资源管理模块,引入Docker容器虚拟化技术对多终端资源进行虚拟化,解决了终端资源异构性问题,形成按需使用的资源池,以容器的形式提供服务和进行管理。服务管理模块,包括服务注册、服务节点选择、服务生命周期管理等功能。任务调度模块,可根据不同任务请求对资源的消耗情况,调度合适的节点进行执行。此外设计了弹性服务模块,预测用户服务请求变化趋势,调节终端服务规模。并提出了去中心化的自组织网络结构,实现多终端节点管理。
% 针对终端设备资源分布较为分散、资源异构性强、不容易管理和利用等问题,本文设计了基于容器化多终端协同服务系统。在所提出的基于容器化多终端协同服务系统中,设计了资源管理模块,引入Docker容器虚拟化技术对多终端资源进行虚拟化,克服终端资源异构性,形成按需使用的资源池,以容器的形式进行利用和管理。参考微服务架构,设计了服务管理模块,包括服务注册、服务节点选择、服务生命周期管理等。设计了任务调度模块,根据不同任务请求对资源的消耗情况,调度合适的节点进行执行。设计了弹性服务模块,预测用户服务请求变化趋势,调节终端服务规模。提出了去中心化的自组织网络结构,对多终端节点进行管理。

(2)基于容器的多终端透明计算迁移技术

针对终端服务中的服务响应时间长的问题,提出一种基于容器的Web Worker透明计算迁移技术。通过对终端底层Web应用执行环境中的部分接口进行修改,通过WebSocket通信机制,将Web应用中的Web Worker迁移到部署在边缘容器集群中的服务端进行执行。实验结果表明,使用基于容器的多终端透明计算迁移技术能够在Web Worker数量较多的情况下,减少Web应用的总执行时间,最高能够减少80.6\%,对于提高终端用户体验有着明显的效果;相比非透明计算迁移技术,基于容器的多终端透明计算迁移技术最高能够减少应用执行时间33.1\%。

(3)多终端协同服务任务调度算法 

为了解决多终端协同服务任务调度问题,合理分配终端资源,降低终端服务费用,减少任务执行时间,提高用户体验,提出一种带有随机跳出机制的动态权重蝗虫优化算法(DJGOA)。DJGOA算法在元启发式算法蝗虫优化算法的基础上增加了基于完全随机跳出因素的跳出机制和动态的权重参数。测试实验结果表明所提出的带有随机跳出机制的动态权重蝗虫优化算法在寻找最优结果方面具有较好的效果。为了更进一步优化算法的性能,基于DJGOA算法,提出了一种改进蝗虫优化算法(IGOA)。
IGOA算法引入了变型的sigmoid函数作为非线性舒适区调节参数,并结合了基于$L\acute{e}vy$飞行的局部搜索机制和基于线性递减参数的随机跳出策略。
仿真实验结果表明,改进蝗虫优化算法在解决多终端协同服务任务调度问题上也能够取得较好的效果,相比其他对比算法效果最高提升31.9\%。

(4)基于预测的容器弹性服务策略

为了解决多终端协同服务技术中的预部署问题,促进提高终端资源利用率和降低用户服务请求响应等待时间之间的平衡,提出了基于预测的容器弹性服务策略。提出了一种改进卡尔曼滤波算法用于预测用户服务请求趋势。根据预测结果弹性部署容器服务,动态调整多终端协同服务的规模,平衡提高终端资源利用率与降低用户服务请求响应等待时间之间的关系。

\keywords{容器虚拟化,智能终端,协同服务,计算迁移,任务调度,弹性服务}% 中文关键词
%-
%-> 英文摘要
%-
\chapter*{Abstract}\chaptermark{Abstract}% 摘要标题

With the rapid development of edge computing technology, user terminal devices at the edge of the network are playing an increasingly important role in people's digital life.
User terminal devices need to undertake more and more computing tasks, which also puts higher and higher requirements on the computing power of terminal devices.
Commonly used cloud computing solutions can bring problems such as poor real-time performance, high deployment cost, and bandwidth congestion in the scenario of user terminal device services. It is possible to use the idle computing resources on the terminal devices close to the users to provide users with nearby computing services. 

Due to the scattered resource distribution and heterogeneous resource heterogeneity of multiple terminals, there are problems such as difficulty in resource utilization, low utilization rate, long service response time, and poor user experience. In order to improve the utilization of terminal resources and enhance the user experience, this paper has carried out research on containerized multi-terminal collaborative service technology.
This paper first introduces container technology to solve the heterogeneous and management problems of terminal resources, and designs a multi-terminal collaborative service system based on containerization. Furthermore, this paper proposes a container-based multi-terminal transparent computing migration technology to reduce service response time and improve user experience. Then, this paper carries out the optimization of multi-terminal collaborative service task scheduling algorithm, which rationally allocates terminal resources, reduces terminal service costs, and reduces service response time. Finally, this paper studies the terminal elastic service technology based on prediction and improves the utilization of terminal resources.

The main research contents and results of this paper are as follows:

(1) Architecture design based on containerized multi-terminal service system

The containerized multi-terminal collaborative service system and resource management module are designed. The Docker container virtualization technology is introduced to virtualize multi-terminal resources, which solves the heterogeneity of terminal resources and forms an on-demand resource pool to provide service and management in the form of containers. Service management module is proposed, including service registration, service node selection, service lifecycle management and other functions. The task scheduling module can schedule the tasks to the appropriate nodes according to the resource consumption of the different task requests. In addition, an elastic service module is designed to predict the trend of user service requests and adjust the scale of terminal services. A decentralized self-organizing network structure is proposed to realize multi-terminal node management in this chapter.

(2) Container-based multi-terminal transparent computing offloading technology

Aiming at the problem of long service response time in terminal service, a container-based Web Worker transparent computing offloading technology is proposed. By modifying some interfaces in the underlying web application execution environment of the terminal, the Web Workers in the web application are migrated to the server deployed in the edge container cluster for execution through the WebSocket communication mechanism. The experimental results show that the container-based multi-terminal transparent computing offloading technology can reduce the total execution time of the web application by 80.6\% in the case of a large number of Web Workers, which has obvious effects on improving the end user experience. Compared to the non-transparent computing offloading technology, container-based multi-terminal transparent computing offloading technology can reduce application execution time by 33.1\%.

(3) Multi-terminal collaborative service task scheduling algorithm

In order to solve the multi-terminal collaborative service task scheduling problem, allocate terminal resources reasonably, reduce terminal service cost, reduce task execution time and improve user experience, a dynamic weight grasshopper optimization algorithm with random jumping (DJGOA) is proposed. Based on the meta-heuristic algorithm grasshopper optimization algorithm, the DJGOA algorithm adds a jumping mechanism based on a completely random jumping factor and a dynamic weight parameter. The experimental results show that the proposed dynamic weight grasshopper optimization algorithm with random jumping has a good effect in finding the optimal result. In order to further optimize the performance of the algorithm, an improved grasshopper optimization algorithm (IGOA) is proposed based on the DJGOA algorithm.
The IGOA algorithm introduces a variant sigmoid function as a nonlinear comfort zone adjustment parameter, and combines a local search mechanism based on $L\acute{e}vy$ flight and a random jumping strategy based on linear descending parameters.
The simulation results show that the improved grasshopper optimization algorithm can achieve better results in solving the multi-terminal collaborative service task scheduling problem, and the effect is up to 31.9\% compared with other comparison algorithms.

(4) Prediction-based  elastic container service strategy

In order to solve the pre-deployment problem in multi-terminal collaborative service technology, promote the balance between the terminal resource utilization and the reduction of user service request response waiting time, a prediction-based elastic container service strategy is proposed. An improved Kalman filter algorithm is proposed to predict the trend of user service requests. According to the prediction result, the container service is flexibly deployed, the scale of the multi-terminal collaborative service is dynamically adjusted, and the relationship between the utilization of the terminal resource and the reduction of waiting time of the user service request is balanced.


\KEYWORDS{Container Virtualization, Intelligent Terminal, Collaborative Service, Computing Offloading, Task Scheduling, Elastic Service}% 英文关键词
%---------------------------------------------------------------------------%
