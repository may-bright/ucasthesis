\chapter[致谢]{致\quad 谢}\chaptermark{致\quad 谢}% syntax: \chapter[目录]{标题}\chaptermark{页眉}
\thispagestyle{noheaderstyle}% 如果需要移除当前页的页眉
%\pagestyle{noheaderstyle}% 如果需要移除整章的页眉


感谢我的导师倪宏老师这五年来对我的指导与帮助。倪老师无论平时工作多么繁忙都会抽出时间指导我的科研工作,在我进行论文选题、学术写作、博士开题答辩、博士中期答辩等工作的过程中,倪老师为我提出了很多宝贵的意见与建议,倪老师认真负责的做事风格、严谨的治学态度、高瞻远瞩的学术视野,都让我受益匪浅。

感谢我的导师郭志川老师,在我攻读博士学位期间对我的整体科研方向以及具体的研究内容都进行了认真细致的指导,并在我学术论文写作和修改的过程中为我提供了很多帮助。在研究组内的科研工作中,郭老师也带领我参与了很多科研项目的立项和研究的过程,让我在求学之路上收获颇丰。

感谢国家网络新媒体工程技术研究中心主任王劲林研究员。王老师总是能够深入浅出地分析学术问题,为我指点迷津,对我的论文选题工作给予了很多的帮助。在中心学习的这几年,王老师渊博的学识、认真务实的科研态度、勤奋的工作作风、广阔的学术视野,都给我留下了深刻的印象,值得我好好学习。
王老师不仅关心每一位同学的科研进展,更关心大家的身心健康。在我读博期间遇到来自学习和生活的巨大压力的时候,是王老师通过积极的沟通、主动的交流、耐心的倾听和认真的指导帮助我走出心理困境,坦然面对压力,顺利完成学业。

感谢中心副主任邓浩江研究员。邓老师在我保研面试的时候为我介绍中心的情况,让我对这个实验室第一次有了比较详细的了解,并且在面试中给我打了不错的分数,让我有机会进入这个实验室完成博士阶段的学习和工作。同时也感谢邓老师在我读博期间对我研究方向及论文的指导以及对我生活和身体健康方面的关心。

感谢终端研究室主任朱小勇老师。在中心的工程工作与科研工作中,接触最多的就是项目组长朱老师了。感谢朱老师带着我参与了很多工程项目。从安徽广电智能终端项目到欧洲媒体服务支撑平台项目,从北京到深圳,从DSP二楼的终端组到DSP五楼的小黑屋,朱老师用认真负责的工作态度、优秀的技术能力、丰富的工程经验以及无比的耐心与包容,帮助我从编程小菜鸟成长为可以独当一面的工程骨干。那些跟着朱师兄一起做项目、一起出差、一起熬夜赶项目节点、一起开会讨论的日子,是我迅速收获、迅速成长的时光,也是我读博阶段非常难忘的经历。也感谢朱老师在科研项目中对我的指导与帮助。无论是嵌入式容器的研究,还是多终端协同服务技术的研究,朱老师都能够帮助我理清研究的思路,选择合适的技术路线,并为我的研究工作提出了很多建设性意见,帮助我完成相关的论文和专利。

感谢终端组韩锐老师、胡琳琳老师、周学志老师、马凤华老师、刘春梅老师对我的科研工作与工程工作的指导与帮助。

感谢的程钢老师、叶晓舟老师、曾学文老师、陈晓老师、王玲芳老师、刘学老师、盛益强老师、脱立恒老师、田野老师、冯丽茹老师、姜艳老师、陈君老师、冯新老师、任晓青老师、徐莉老师等老师在我读博期间对我的指导与帮助。

感谢中心主管学生工作的卢美英老师、尤佳莉老师、杨中臻师兄这五年来对我们的生活方面与身心健康方面的关心与照顾。

感谢终端组同学。

感谢中心的师兄师姐师弟师妹。

感谢新媒体14级同学。

感谢国科大电子学院6班。

特别感谢王旭大师兄。

特别感谢室友伍洪桥。

特别感谢张恩泽、韩智敏。

最后感谢我的父母家人。


\chapter{作者简历及攻读学位期间发表的学术论文与研究成果}

% \textbf{本科生无需此部分}。

\section*{作者简历}

赵然,男,内蒙古通辽人,1992年出生,中国科学院声学研究所博士研究生。

2010年08月——2014年07月,在清华大学自动化系获得学士学位。

2014年09月——2019年07月,在中国科学院声学研究所攻读博士学位。


\section*{已发表(或正式接受)的学术论文:}

[1] Ran Zhao, Hong Ni, Hangwei Feng, Yaqin Song, Xiaoyong Zhu. An Improved Grasshopper Optimization Algorithm for Task Scheduling Problems[J]. International Journal of Innovative Computing, Information and Control, 2019(EI期刊,已录用)

[2] Ran Zhao, Hangwei Feng, Xiaoyong Zhu, Hong Ni. A Dynamic Weight Grasshopper Optimization Algorithm with Random Jumping[C]. 2018 3rd International Conference on Computer, Communication and Computational Sciences (IC4S2018),2018 (EI会议,已录用)

[3] 赵然,朱小勇.微服务架构评述[J].网络新媒体技术,2019,8(01):58-61+65. (核心期刊,已发表)

\section*{申请或已获得的专利:}

1.朱小勇,赵然,一种微服务故障检测处理方法及装置(申请号: 201711368632.3)

2.朱小勇,常乐,郭志川,赵然,一种Docker容器多进程管理方法及系统(申请号:201611090130.4 )

3. 朱小勇,邓丽君,郭志川,赵然,常乐,一种基于嵌入式系统的RancherOS ros核心模块移植方法(申请号:201611069177.2)

\section*{软件著作权:}

1.朱小勇,赵然,视频点播平台客户端软件. 登记号:2017SRBJ0226

2.韩锐、李超、赵然、朱小勇、郭志川,机顶盒IP升级软件. 登记号:2017SRBJ0116

\section*{参加的研究项目及获奖情况:}

1.中国科学院声学研究所率先行动“端到端虚拟化关键技术研究与系统开发”(编号: Y654101601)

2.中国科学院声学研究所青年英才“嵌入式容器文件系统关键技术研究”(编号:QNYC201714)

3.中国科学院先导专项课题:SEANET技术标准化研究与系统研制(编号:XDC02010801) 

4.欧洲媒体服务支撑平台项目

5.安徽广电智能终端项目


\cleardoublepage[plain]% 让文档总是结束于偶数页,可根据需要设定页眉页脚样式,如 [noheaderstyle]

