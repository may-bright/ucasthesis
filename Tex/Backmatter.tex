\chapter[致谢]{致\quad 谢}\chaptermark{致\quad 谢}% syntax: \chapter[目录]{标题}\chaptermark{页眉}
\thispagestyle{noheaderstyle}% 如果需要移除当前页的页眉
%\pagestyle{noheaderstyle}% 如果需要移除整章的页眉


% 感谢我的导师倪宏研究员。

% 感谢我的导师郭志川研究员。

% 感谢实验室主任王劲林研究员。

% 感谢实验室副主任邓浩江研究院。

% 感谢终端研究室主任朱小勇老师。

% 感谢终端组韩锐、胡琳琳、周学志、马凤华、刘春梅等老师。

% 感谢


\chapter{作者简历及攻读学位期间发表的学术论文与研究成果}

% \textbf{本科生无需此部分}。

\section*{作者简历}

赵然,男,内蒙古通辽人,1992年出生,中国科学院声学研究所博士研究生。

2010年08月——2014年07月,在清华大学自动化系获得学士学位。

2014年09月——2019年07月,在中国科学院声学研究所攻读博士学位。


\section*{已发表(或正式接受)的学术论文:}

[1] Ran Zhao, Hong Ni, Hangwei Feng, Yaqin Song, Xiaoyong Zhu. An Improved Grasshopper Optimization Algorithm for Task Scheduling Problems[J]. International Journal of Innovative Computing, Information and Control, 2019(EI期刊,已录用)

[2] Ran Zhao, Hangwei Feng, Xiaoyong Zhu, Hong Ni. A Dynamic Weight Grasshopper Optimization Algorithm with Random Jumping[C]. 2018 3rd International Conference on Computer, Communication and Computational Sciences (IC4S2018),2018 (EI会议,已录用)

[3] 赵然,朱小勇.微服务架构评述[J].网络新媒体技术,2019,8(01):58-61+65. (核心期刊,已发表)

\section*{申请或已获得的专利:}

1.朱小勇,赵然,一种微服务故障检测处理方法及装置(申请号: 201711368632.3)

2.朱小勇,常乐,郭志川,赵然,一种Docker容器多进程管理方法及系统(申请号:201611090130.4 )

3. 朱小勇,邓丽君,郭志川,赵然,常乐,一种基于嵌入式系统的RancherOS ros核心模块移植方法(申请号:201611069177.2)

\section*{软件著作权:}

1.朱小勇,赵然,视频点播平台客户端软件. 登记号:2017SRBJ0226

2.韩锐、李超、赵然、朱小勇、郭志川,机顶盒IP升级软件. 登记号:2017SRBJ0116

\section*{参加的研究项目及获奖情况:}

1.中国科学院声学研究所率先行动“端到端虚拟化关键技术研究与系统开发”(编号: Y654101601)

2.中国科学院声学研究所青年英才“嵌入式容器文件系统关键技术研究”(编号:QNYC201714)

3.中国科学院先导专项课题:SEANET技术标准化研究与系统研制(编号:XDC02010801) 

4.欧洲媒体服务支撑平台项目

5.安徽广电智能终端项目


\cleardoublepage[plain]% 让文档总是结束于偶数页,可根据需要设定页眉页脚样式,如 [noheaderstyle]

