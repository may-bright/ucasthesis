\chapter[致谢]{致\quad 谢}\chaptermark{致\quad 谢}% syntax: \chapter[目录]{标题}\chaptermark{页眉}
\thispagestyle{noheaderstyle}% 如果需要移除当前页的页眉
%\pagestyle{noheaderstyle}% 如果需要移除整章的页眉

时光如白驹过隙,一转眼走到了二十载求学生涯的最终章。曾经无数次幻想,在毕业论文的收尾工作里,要如何来书写致谢这一章,然而当这一天真的到来的时候,却又迟迟不敢动笔。一则是因为一路走来,要感谢的人很多,老师、同学、朋友、家人,万千回忆凝聚于笔尖,竟不知从何处着墨。二则是因为总是害怕这一章写完,就真的要同自己的学生时代告别了。一直拖到今天,望着窗外凌晨4点的北京,听着陈绮贞的《送别》的歌声,终于还是决定开始写了。

首先要感谢我的导师倪宏老师这五年来对我的指导与帮助。倪老师无论平时工作多么繁忙都会抽出时间指导我的科研工作,在我进行论文选题、学术写作、博士开题答辩、博士中期答辩等工作的过程中,倪老师为我提出了很多宝贵的意见与建议,倪老师认真负责的做事风格、严谨的治学态度、高瞻远瞩的学术视野,都让我受益匪浅。

感谢我的导师郭志川老师,在我攻读博士学位期间对我的整体科研方向以及具体的研究内容都进行了认真细致的指导,并在我学术论文写作和修改的过程中为我提供了很多帮助。在研究组内的科研工作中,郭老师也带领我参与了很多科研项目的立项和研究的过程,让我在求学之路上收获颇丰。

感谢国家网络新媒体工程技术研究中心主任王劲林研究员。王老师总是能够深入浅出地分析学术问题,为我指点迷津,对我的论文选题工作给予了很多的帮助。在中心学习的这几年,王老师渊博的学识、认真务实的科研态度、勤奋的工作作风、广阔的学术视野,都给我留下了深刻的印象,值得我好好学习。
王老师不仅关心每一位同学的科研进展,更关心大家的身心健康。在我读博期间遇到来自学习和生活的巨大压力的时候,是王老师通过积极的沟通、主动的交流、耐心的倾听和认真的指导帮助我走出心理困境,坦然面对压力,顺利完成学业。

感谢中心副主任邓浩江研究员。邓老师在保研面试中让我有机会进入这个实验室完成博士阶段的学习和工作。也感谢邓老师在我读博期间对我研究方向及论文的指导以及对我生活和身体健康方面的关心。

感谢终端研究室主任朱小勇老师。在中心的工程工作与科研工作中,接触最多的就是项目组长朱老师了。感谢朱老师带着我参与了很多工程项目。从安徽广电智能终端项目到欧洲媒体服务支撑平台项目,从北京到深圳,从DSP二楼的终端组到DSP五楼的小黑屋,朱老师用认真负责的工作态度、优秀的技术能力、丰富的工程经验以及无比的耐心与包容,帮助我从编程小菜鸟成长为可以独当一面的工程骨干。那些跟着朱师兄一起做项目、一起出差、一起熬夜赶项目节点、一起开会讨论的日子,是我迅速收获、迅速成长的时光,也是我读博阶段非常难忘的经历。也感谢朱老师在科研项目中对我的指导与帮助。无论是嵌入式容器的研究,还是多终端协同服务技术的研究,朱老师都能够帮助我理清研究的思路,选择合适的技术路线,并为我的研究工作提出了很多建设性意见,帮助我完成相关的论文和专利。

感谢终端组韩锐老师、胡琳琳老师、周学志老师、马凤华老师、刘春梅老师对我的科研工作与工程工作的指导与帮助。

感谢的程钢老师、叶晓舟老师、曾学文老师、陈晓老师、王玲芳老师、刘学老师、盛益强老师、脱立恒老师、田野老师、冯丽茹老师、姜艳老师、陈君老师、冯新老师、任晓青老师、杨静漪老师、徐莉老师等在我读博期间对我的指导与帮助。

感谢中心主管学生工作的卢美英老师、尤佳莉老师、杨中臻师兄这五年来对我们的生活方面与身心健康方面的关心与照顾。

感谢中心的吴京洪师兄、董海韬师兄、樊浩师兄、卓煜师姐、黄河师兄、曹作伟师兄、李超鹏师兄在我初入实验室、论文写作、论文投稿、找工作、准备毕业等过程中提供的宝贵经验与帮助,让我少走了很多弯路。

感谢终端组的王旭师兄、黄兴旺师兄、肖伟民师兄、耿筱林师姐、耿立宏师兄、刘丽琴师姐、陈霄师弟、李超师弟、马博韬师弟、宋锐星师妹、包沙如拉师妹、王慧鑫师妹、宋雅琴师妹、刘轶峰师弟、董翰泽师弟、冯航伟师弟、张晶师妹、刘彤师妹、方立师弟、段英杰师妹、许勇师弟、杨丹师妹、晁一超师弟。忘不了组里那些朝夕相处的日子:新人破冰时紧张的自我介绍,饭后闲聊的欢声笑语,每周组会上的工作交流,元旦的时候一起聊天跨年,找工作时候的互相帮助,科研过程中的学术讨论、迷茫无助时候的陪伴与关心。那些充满汗水与欢笑的时光,一点一滴,组成了我读博生活的日常。

% 感谢国科大电子学院6班的同学们,尤其是战鸽、刘经纬、高翔、蔡宁、胡中韬、刘科栋、高大亮、石伟志、王全东、齐晓琳。

特别感谢新媒体14级一起求学的同窗好友麻朴方、许丹凤、田娟娟、伍洪桥、姜凯华、薛寒星、廖怡、常乐、李强、黎江源、邓丽君、郑抗、贾正义、唐志斌、王昭、唐政治。这五年来经历的那些当时只道是寻常的时光,那些本以为长得看不到尽头的日子,都成为我最闪光的回忆。还没说再见,已经开始怀念。怀念在怀柔一起滑雪一起爬山的日子,怀念跨年夜的彻夜长谈,怀念一起复习准备期末考试的时光,怀念答辩前夜的并肩作战,怀念最艰难的时候大家的互相鼓励和加油打气。感谢你们让我知道,我不是一个人在战斗,成长之路上有你们的陪伴,真好!

特别感谢王旭大师兄,感谢你在我最低谷的时候花时间陪伴我,耐心地听我倾诉那些傻瓜的念头,认真地开导我帮助我,让我逃离那个梦魇的吞噬,顺利完成学业。

特别感谢伍洪桥同学。从清华到果壳,从紫荆公寓到青年公寓,从五道口到雁栖湖,从佳木斯到垦丁,都留下了我们友谊的见证。九年同学,七年室友,一起吃饭,一起打球,一起游戏玩耍,一起学习成长,感谢你多年以来的陪伴与帮助。

特别感谢一生的挚友张恩泽和韩智敏。感谢你们多年以来做我坚实的后盾和力量的源泉。虽然远隔千山万水,但是我知道你们一直在我身边。

特别感谢我的父母,感谢你们给了我健康的身体和快乐的成长环境,让我能够成长成为今天的自己。感谢你们二十多年来在背后默默地付出,给我无条件的支持与信任,在我得意的时候给我警示,在我失意的时候给我安慰与鼓励,你们永远是我前进的动力,谢谢你们!

最后,感谢每一位陪我走过这段路的人们,千言万语,锦书遥寄,长亭送别,相顾依依。

\begin{flushright}

    赵然

    2019年4月于北京


\end{flushright}


\chapter{作者简历及攻读学位期间发表的学术论文与研究成果}

% \textbf{本科生无需此部分}。

\section*{作者简历}

赵然,男,内蒙古通辽人,1992年出生,中国科学院声学研究所博士研究生。

2010年08月——2014年07月,在清华大学自动化系获得学士学位。

2014年09月——2019年07月,在中国科学院声学研究所攻读博士学位。


\section*{已发表(或正式接受)的学术论文:}

[1] Ran Zhao, Hong Ni, Hangwei Feng, Yaqin Song, Xiaoyong Zhu. An Improved Grasshopper Optimization Algorithm for Task Scheduling Problems[J]. International Journal of Innovative Computing, Information and Control, 2019(EI期刊,已录用)

[2] Ran Zhao, Hong Ni, Hangwei Feng, Xiaoyong Zhu. A Dynamic Weight Grasshopper Optimization Algorithm with Random Jumping[C]. 2018 3rd International Conference on Computer, Communication and Computational Sciences (IC4S2018),2018 (EI会议,已录用)

[3] 赵然,朱小勇.微服务架构评述[J].网络新媒体技术,2019,8(01):58-61+65.(核心期刊,已发表)

[4] 赵然,郭志川,朱小勇.基于容器的Web Worker透明边缘计算迁移系统[J].微电子学与计算机.(核心期刊,在投)

[5] 赵然,郭志川,朱小勇.基于卡尔曼滤波预测的容器弹性服务策略[J].计算机与现代化.(核心期刊,在投)

\section*{申请或已获得的专利:}

1.朱小勇,赵然,一种微服务故障检测处理方法及装置. 申请号:201711368632.3

2.朱小勇,常乐,郭志川,赵然,一种Docker容器多进程管理方法及系统. 申请号:201611090130.4

3. 朱小勇,邓丽君,郭志川,赵然,常乐,一种基于嵌入式系统的RancherOS ros核心模块移植方法. 申请号:201611069177.2

\section*{软件著作权:}

1.朱小勇,赵然,视频点播平台客户端软件.登记号:2017SRBJ0226

2.韩锐、李超、赵然、朱小勇、郭志川,机顶盒IP升级软件. 登记号:2017SRBJ0116

\section*{参加的研究项目:}

1.中国科学院声学研究所率先行动:端到端虚拟化关键技术研究与系统开发. 编号:Y654101601

2.中国科学院声学研究所青年英才:嵌入式容器文件系统关键技术研究(编号:QNYC201714)

3.中国科学院先导专项课题:SEANET技术标准化研究与系统研制(编号:XDC02010801)

4.欧洲媒体服务支撑平台项目

5.国家科技支撑计划课题:电视商务综合体新业态应用示范(编号:2012BAH73F02)

\section*{获奖情况:}
2016-2017学年,获中国科学院大学三好学生

\cleardoublepage[plain]% 让文档总是结束于偶数页,可根据需要设定页眉页脚样式,如 [noheaderstyle]

