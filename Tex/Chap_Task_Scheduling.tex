\chapter{资源受限终端任务调度策略}\label{chap:task_scheduling}

\section{引言}

任务调度问题是指系统在同时接收到多个服务请求任务的时候,将这些任务合理地分配给多个智能终端上的容器进行处理,由于不同的容器所在智能终端的处理能力不同,每个容器所占用的资源也不同,导致不同的调度方案结果会有差异,需要针对在容器化智能终端协同服务场景下的一些特点来进行取舍和进一步的优化,以追求对于终端资源利用的最大化。在该任务场景下,最常见的用户需求就是实时性需求,也即要求任务能够被快速响应、快速执行、且执行结果能够快速回传给用户,因此最小化任务完成时间是任务调度问题最主要的目标。除此之外,需要考虑的因素还涉及硬件设备功耗、分布式设备负载均衡度等。

\section{相关工作}

\subsection{传统任务调度问题求解方法}

任务调度问题是将多个任务计划到约束下的多个节点的问题。任务调度问题可能是一个优化问题。应用多种算法来解决任务调度问题。基于最佳资源选择 (BRS) 的算法, 如最大最小、最小、苦难等, 是解决任务调度问题的传统方法。一些元启发式算法, 如 PSO 和基于 pso 的改进算法, 是处理任务调度问题的新方法。

\subsection{启发式算法求解方法}

近年来, 对任务调度问题进行了大量的研究。随着研究的进行, 许多元启发式算法被用来处理复杂的优化问题。元启发式算法具有简单的操作和较少的开销, 能够找到全局最优。

许多元启发式算法被引入到优化问题中。GA 是戈德伯格在1988年提出的经典元启发式算法, 它将自然选择理论引入到优化过程中, 包括突变、交叉和选择 \ citefonseca1995an,Whitley1994, Tanese1989DGA, Ng广州市2018。虽然 GA 的性能相当好, 但遗传算法的操作过于复杂, 无法实现, 不适合某些情况。一些元启发式算法的灵感来自昆虫、鱼类、鸟类和其他群体生物的自然行为。粒子群算法 (PSO) 是肯尼迪1995年提出的经典元启发式算法。PSO 的原理简单, 性能显著 \ 1995年的柠檬素颗粒, Liao2007, Gomathi2013。蚁群优化算法 (ACO) 是受蚂蚁在鸟巢和食物来源之间自然觅食行为的启发。ACO 利用化学信息素在中国的人群中进行交流。

最近提出了一些新的元启发式优化算法。关于改进这些算法的研究并不多。蚂蚁狮子优化器提出 (ALO) 在2015年是受狮子 \ citemirjilili2013 蚂蚁的狩猎行为的启发。鲸鱼优化算法 (WOA) 于2016年提出。WA 模拟鲸鱼的自然狩猎行为。2016年提出的蜻蜓算法 (DA) 是受蜻蜓群在自然界中的静态和动态行为的启发。


2017年, Shahrzad Saremi 和 Seyedali Mirjalili 提出了一种新的元启发式优化算法, 称为蝗虫优化算法 (GOA)。GOA 利用群体内部的相互作用和蜂群外的风的影响来模拟蝗虫群的迁徙行为, 寻找目标食物 \ citesare2013 2017 蝗虫。GOA 算法利用群智能, 通过在蝗虫群之间分享经验, 确定搜索方向, 找到最佳或近似的最佳位置。GOA 还使用了具有多个迭代的进化方法, 以提高群智能的效率。

开发了一些基于 GOA 的改进算法。OBLGOA 是由艾哈迈德·埃维斯在2018年提出的 \ ceewees 2013 改进。根据目前的搜索位置, 引入了基于对立面的学习策略, 以生成相反的解作为候选方案。OBL 策略可以提高算法的收敛速度, 但由于缺乏随机性, 改进有限。桑卡拉阿罗拉提出了混沌蝗虫优化算法在 2018年 \ 柠檬酸乱糟糟。将混沌映射应用到算法中, 提高了 GOA 的性能。采用10幅混沌映射来评价混沌理论的影响。结果并不是特别理想, 因为混沌因素在处理许多基准函数时并不合适。提出了一种基于 GOA 的新算法来解决优化问题和任务调度问题。

\section{GOA算法}

蝗虫优化算法模拟蝗虫的昆虫群行为。蝗虫蜂拥而至, 远距离迁徙, 寻找一个有食物的新栖息地。在这个过程中, 蝗虫之间的互动在蜂群内部互相影响。风的力量和蜂群外的重力影响蝗虫的轨迹。食物的目标也是一个重要的影响因素。

受上述三个因素的影响, 移民过程分为勘探和开发两个阶段。在勘探阶段, 鼓励蝗虫快速、突然地移动, 寻找更多潜在的目标区域。在开发阶段, 蝗虫往往在当地移动, 以寻找更好的目标地区。蝗虫自然实现了勘探开发寻找食物来源的两种迁徙趋势。此过程可以抽象为优化问题。蝗虫群被抽象为一群搜索代理。

Seyedali Mirjalili 在文献[]中提出了蝗虫群体迁移的数学模型。具体的模拟公式如下:

\begin{equation}
    X_i = S_i + G_i + A_i 
\end{equation}

这里变量$X_i$是第i个搜索单位的位置,变量$S_i$代表蝗虫集群内部搜索单位间社会交互对第i个搜索单位的影响因子,变量$G_i$代表蝗虫集群外部重力因素对第i个搜索单位的影响因子,变量$A_i$代表风力的影响因子。变量$S_i$的定义公式如下:

\begin{equation}
    S_i = \sum_{j=1, j\neq{i}}^N s(d_{ij})\widehat{d_{ij}}
\end{equation}

这里变量$d_{ij}$代表第i个搜索单位和第j个搜索单位之间的欧式距离,计算方法如下:
\begin{equation}
    d_{ij}=|x_j-x_i|
\end{equation}

变量$\widehat{d_{ij}}$代表第i个搜索单位和第j个搜索单位之间的单位向量,计算方法如下:

\begin{equation}
    \widehat{d_{ij}}=\frac{x_j-x_i}{d_{ij}}
\end{equation}

\emph{s}是一个函数,用于计算蝗虫集群之间的社会关系影响因子,该函数定义如下:

\begin{equation}
    s(r) = fe^{\frac{-r}{l}}-e^{-r}
\end{equation}

这里\emph{e}是自然底数,变量\emph{f}代表吸引力因子,参数\emph{l}代表吸引力长度。
% where \emph{e} is the Natural Logarithm, \emph{f} represents the concentration of attraction and the parameter of \emph{l} shows the attractive length scale.
当应用于解决数学优化问题的时候,为了优化数学模型,公式1中需要加入一些适当的改动。代表集群外部影响因子的变量$G_i$和$A_i$需要被替换为目标食物的位置。这样计算公式就变成了如下的样子:

\begin{equation}
    x_i = c \Bigl(\sum_{j=1,j\neq{i}}^N c \frac{u-l}{2}s (\lvert x_j-x_i \rvert )\frac{x_j-x_i}{d_{ij}} \Bigr) + \widehat{T_d}
\end{equation}

这里参数\emph{u}和参数\emph{l}分别代表搜索空间的上界和下界。变量$\widehat{T_d}$是目标食物的位置,在优化问题的数学模型中代表所有搜索单位在整个搜索过程中所能找到的最优的解的位置。另外,参数\emph{c}是搜索单元的搜索舒适区控制参数,改变参数\emph{c}的大小可以平衡搜索过程中的“开拓”和“探索”两个阶段。参数\emph{c}的计算方式如下:

\begin{equation}
    c = cmax - iter \frac{cmax - cmin}{MaxIteration}
\end{equation}


这里参数\emph{cmax}和参数\emph{cmin}分别是参数\emph{c}的最大值和最小值,参数\emph{iter}代表当前的迭代次数,参数\emph{MaxIteration}代表最大迭代次数。

在优化问题的求解过程中,公式4作为演进公式,被不断循环迭代来寻找最优解,直到达到迭代终止条件为止。通常迭代终止条件为达到预设的最大迭代次数,或者所得到的最优解满足预设的最优解条件。在本研究所涉及到的优化问题中,迭代终止条件均为达到预设的最大迭代次数。在迭代演进的过程结束后,该算法可以得到一个近似的最优解的位置以及相应的最优解的值。
\section{Dynamic Weight GOA with Random Jumping(DJGOA)}

\section{Improved GOA(IGOA)}

\section{任务调度问题}
\subsection{问题描述}
\subsection{任务调度模型}
\subsection{实验结果}


\section{本章小结}