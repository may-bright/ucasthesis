\chapter{总结与展望 }\label{chap:summary_and_future_work}

\section{工作总结}

随着计算机技术的快速发展以及边缘计算技术的逐渐成熟,位于网络边缘的用户终端设备在人们的数字化生活中正扮演着越来越重要的角色。用户终端需要承担越来越多的计算任务,这也对终端设备的计算能力提出了越来越高的要求。相比云计算技术可能会带来实时性差、部署成本高、带宽拥挤等问题,利用靠近用户的设备上空闲的计算资源协同为用户终端提供就近的计算服务成为一种可行的解决方案。

基于终端设备资源的分散性、异构性等特点,如何将边缘终端设备的空闲资源管理起来并利用多终端设备协同提供服务以达到提高终端资源利用效率、提高终端用户体验的目的,成为本文需要重点解决的问题。为此本文开展基于容器化多终端协同服务技术研究。本文首先介绍了云计算技术、容器虚拟化技术、计算迁移技术、任务调度算法等相关技术和算法的研究现状,分析其目前还存在的问题。本文结合容器虚拟化技术和多终端协同服务技术,设计了基于容器化的多终端协同服务系统,并主要研究系统中的基于容器的多终端透明计算迁移技术、多终端协同服务任务调度算法优化以及基于预测的终端弹性服务技术。

本文具体研究内容与成果如下:

1.基于容器化多终端服务系统架构设计

针对终端设备资源分布较为分散、资源异构性强、不容易管理和利用等问题,设计了基于容器化多终端协同服务系统。在所提出的基于容器化多终端协同服务系统中,设计了资源管理模块,引入Docker容器虚拟化技术对多终端资源进行虚拟化,克服终端资源异构性,形成资源池,可以被上层终端服务根据对资源的类型和数量按需使用。除了对终端资源以容器的形式进行管理和利用,资源管理模块还需要对容器的生命周期进行管理。参考微服务架构,设计了服务管理模块,包括服务之间通过REST的轻量级通信机制进行通信、服务注册、服务节点选择、服务生命周期管理等。设计了任务调度模块,根据不同任务请求对资源的消耗情况,调度合适的节点进行执行。设计了弹性服务模块,统计用户服务请求数量,预测服务请求变化趋势,适当调节终端服务规模。另外还提出了去中心化的自组织网络结构,对多终端节点进行管理。

2.基于容器的多终端透明计算迁移技术

为了将用户周围终端设备的空闲资源利用起来,在对应用保持透明的情况下,为Web应用提供计算迁移服务,提高终端资源利用率,缩短Web应用计算时间,提高终端用户体验,提出一种基于容器的Web Worker透明计算迁移技术。通过对终端底层Web应用执行环境中的部分接口进行修改,将上层Web应用传来的Web Worker创建请求进行翻译并重新封装,通过WebSocket将封装后的请求发送到部署到边缘容器集群中的服务端进行处理。实验结果表明,相比Web应用本地执行,使用基于容器的多终端透明计算迁移技术能够在Web Worker数量较多的情况下,减少Web应用的总执行时间,最高能够减少80.6\%,对于提高终端用户体验有着明显的效果;相比非透明计算迁移技术,基于容器的多终端透明计算迁移技术最高能够减少应用执行时间33.1\%。

3.多终端协同服务任务调度算法 

为了解决多终端协同服务任务调度问题,选择更合适的终端执行终端服务任务,提高终端资源利用率,减少任务执行时间,提高用户体验,基于元启发式算法蝗虫优化算法,提出一种带有随机跳出机制的动态权重蝗虫优化算法。利用基于完全随机跳出因素的跳出机制,提高算法的跳出局部最优的能力;根据搜索阶段的不同,使用动态的权重参数来代替原算法中的线性递减搜索单元权重参数,帮助算法在不同的搜索阶段获得更大的迭代收益。通过一系列benchmark测试函数测试,表明所提出的带有随机跳出机制的动态权重蝗虫优化算法在寻找最优结果方面具有不错的效果。为了更进一步优化算法在寻优问题和任务调度问题上的性能,基于上述改进算法基础,提出了一种改进蝗虫优化算法。引入变型的sigmoid函数作为非线性舒适区调节参数,增强算法的搜索能力;提出基于$L\acute{e}vy$飞行的局部搜索机制,让搜索单元在局部拥有一定的“搜索视觉”,提高算法的局部搜索能力;使用基于线性递减参数的随机跳出策略,增强算法跳出局部最优能力,并将成功跳出的结果影响力维持若干次迭代。通过一系列benchmark测试函数测试,表明所提出的改进蝗虫优化算法,在寻找最优结果方面具有更好的效果。仿真实验结果表明,改进蝗虫优化算法在解决多终端协同服务任务调度问题上也能够取得较好的效果,相比其他对比算法效果最高提升31.9\%。

4.基于预测的容器弹性服务策略

为了解决多终端协同服务技术中的预部署问题,促进提高终端资源利用率和降低用户服务请求响应等待时间之间的平衡,提出了基于预测的容器弹性服务策略。基于卡尔曼滤波器,结合多终端协同服务的特点,提出了一种改进卡尔曼滤波算法。根据预测结果提前弹性部署容器服务,动态调整多终端协同服务的规模,平衡提高终端资源利用率与降低用户服务请求响应等待时间之间的关系。

\section{工作展望}

本文围绕基于容器化的多终端协同服务技术进行研究,进行了基于容器化多终端协同服务系统架构设计、基于容器的多终端透明计算迁移技术、多终端协同服务任务调度算法、基于预测的容器弹性服务策略等问题的研究,并取得了一些研究成果。

但是,本文的研究工作还存在着很多不足,还有着很多需要进行优化和完善的工作可以在未来做进一步的研究。针对第\ref{chap:service_system}章提出的基于容器化多终端协同服务系统架构,需要对提供协同服务的终端设备进行定期健康检查和终端资源状态监测,提供对终端本地服务性能优先保证的策略。针对第\ref{chap:computation_offloading}章提出的基于容器的多终端透明计算迁移技术,需要考虑透明计算迁移的时机,以及研究除了面对Web Worker以外的其他形式的细粒度透明计算迁移技术。针对第\ref{chap:task_scheduling}章提出的多终端协同服务任务调度算法,可以将任务调度模型进一步复杂化,考虑更多的复杂情况,也可以研究含有任务依赖关系的面向DAG图的任务调度算法。针对第\ref{chap:elastic_service}章提出的基于预测的容器弹性服务策略,可以进一步将自动化技术中的控制算法引入进来,对终端服务规模做更加精细的控制和调整。