\chapter{基于预测的容器弹性服务策略 }\label{chap:elastic_service}

\section{1引言}
这一章对应研究点4,弹性服务。如果终端收到服务请求,再启动对应容器进行服务,则启动时间会大大增加用户等待时间。但由于终端资源有限,终端服务不能像云端服务那样一直在后台运行,等待服务请求。因此我们提出基于预测的容器弹性服务策略,根据预测结果提前弹性部署容器服务,平衡二者的关系。
\section{相关工作}
\section{基于预测的容器弹性服务系统设计}
\section{基于卡尔曼滤波的预测算法}
我们利用卡尔曼滤波的方法,采集历史用户请求数据,并对下一时间点的用户请求数量进行预测,弹性调整容器服务规模。
\section{实验结果}
\section{本章小结}