\chapter{绪论}\label{chap:introduction}

\section{研究背景}
随着信息技术的快速发展,人工智能、物联网等技术与智能终端的结合得越来越紧密,在智能终端上除了为用户提供基础服务以外还要提供更智能化的服务。因此,智能终端上要承担的计算任务越来越重。但目前智能终端上所拥有的计算、存储等资源并没有跟上对于智能终端计算能力的需求。与此同时,家庭、办公楼等智能终端环境中还存在着很多相对空闲、计算资源没有得到充分利用的计算设备。越来越重的计算任务与计算资源分配不均的矛盾日益凸显,为了解决这样的矛盾,在整体资源有限的情况下使资源利用达到最优化,我们想到可以使用多个智能终端协同提供服务,提供一种能够整合多终端资源、合理利用多终端资源的服务技术。
\section{研究意义}
本研究针对智能终端计算任务越来越重与终端计算资源分布不均衡的问题,引入容器技术,对于多终端上的资源进行聚合管理,并通过多智能终端协同技术,合理分配利用终端空闲资源,提高智能终端的服务效率以及智能终端的资源利用率,为未来智能终端为用户提供日常生活服务以及与边缘计算、人工智能、物联网等技术的结合打下良好基础,具有重要的研究意义和应用价值。
\section{研究内容}
引入了多智能终端协同技术和容器技术以后,整个系统会存在很多单一终端服务不会遇到的问题,本研究中主要解决其中的四个问题,包括基于容器化多终端服务系统架构设计、基于容器化服务资源提供技术、资源受限终端任务调度策略、基于预测的容器弹性服务策略。
\section{本文内容安排}