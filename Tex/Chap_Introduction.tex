\chapter{绪论}\label{chap:introduction}

\section{研究背景}
随着信息技术的快速发展,人工智能、物联网等技术与智能终端的结合得越来越紧密,在智能终端上除了为用户提供基础服务以外还要提供更智能化的服务。因此,智能终端上要承担的计算任务越来越重。但目前智能终端上所拥有的计算、存储等资源并没有跟上对于智能终端计算能力的需求。与此同时,家庭、办公楼等智能终端环境中还存在着很多相对空闲、计算资源没有得到充分利用的计算设备。越来越重的计算任务与计算资源分配不均的矛盾日益凸显,为了解决这样的矛盾,在整体资源有限的情况下使资源利用达到最优化,我们想到可以使用多个智能终端协同提供服务,提供一种能够整合多终端资源、合理利用多终端资源的服务技术。
\section{研究意义}
本研究针对智能终端计算任务越来越重与终端计算资源分布不均衡的问题,引入容器技术,对于多终端上的资源进行聚合管理,并通过多智能终端协同技术,合理分配利用终端空闲资源,提高智能终端的服务效率以及智能终端的资源利用率,为未来智能终端为用户提供日常生活服务以及与边缘计算、人工智能、物联网等技术的结合打下良好基础,具有重要的研究意义和应用价值。
\section{研究内容}
为了解决终端计算能力跟不上以及终端空闲资源浪费的问题,引入多终端协同服务。引入后会面对如何使用多终端上的空闲资源、提高利用率、管理等问题。
引入了多终端协同技术和容器技术以后,整个系统会存在很多单一终端服务不会遇到的问题,本研究中主要解决其中的四个问题,包括基于容器化多终端服务系统架构设计、基于容器化服务资源提供技术、资源受限终端任务调度策略、基于预测的容器弹性服务策略。

本文针对上面提出的几个问题,首先介绍了边缘计算技术、容器虚拟化技术、计算迁移技术、任务调度算法、预测算法等相关技术和算法的研究现状,分析其目前还存在的问题。本文结合容器虚拟化技术和多终端协同服务技术,设计了基于容器化的多终端协同服务系统,并主要研究系统中的基于容器的多终端透明计算迁移技术、多终端协同服务任务调度算法优化以及基于预测的终端弹性服务技术。

本文具体研究内容如下:
\begin{enumerate}
    \item 引入了多智能终端协同技术和容器技术以后,整个系统会存在很多单一终端服务不会遇到的问题,本研究中主要解决其中的四个问题,包括基于容器化多终端服务系统架构设计、基于容器化服务资源提供技术、资源受限终端任务调度策略、基于预测的容器弹性服务策略。
    \item 引入了多智能终端协同技术和容器技术以后,整个系统会存在很多单一终端服务不会遇到的问题,本研究中主要解决其中的四个问题,包括基于容器化多终端服务系统架构设计、基于容器化服务资源提供技术、资源受限终端任务调度策略、基于预测的容器弹性服务策略。
    \item 为了在终端资源有限且终端资源异构性极强的情况下,合理调度任务请求到更合适的执行节点上,使得任务总体开销更小,减少响应时间,提高用户体验,本研究基于一种群体智能的演进式优化算法————蝗虫优化算法,提出一种带有随机跳出机制的动态权重蝗虫优化算法来解决优化问题。在该算法中,我们在原有蝗虫优化算法的基础上,增加了基于完全随机跳出因素的跳出机制,来提高算法的跳出局部最优的能力。另外,我们还根据搜索阶段的不同,使用动态的权重参数来代替原算法中的线性递减搜索单元权重参数,帮助算法在不同的搜索阶段获得更大的迭代收益。经过一系列测试函数的实验验证,我们提出的带有随机跳出机制的动态权重蝗虫优化算法能够有效提高优化算法的搜索精度及收敛速度。在此基础上,我们进一步提出一种改进蝗虫优化算法,并将其应用于解决多终端任务调度问题。在该算法中,我们
    任务调度问题是指系统在同时接收到多个服务请求任务的时候,将这些任务合理地分配给多个智能终端上的容器进行处理,由于不同的容器所在智能终端的处理能力不同,每个容器所占用的资源也不同,导致不同的调度方案结果会有差异,需要针对在容器化智能终端协同服务场景下的一些特点来进行取舍和进一步的优化,以追求对于终端资源利用的最大化。在该任务场景下,最常见的用户需求就是实时性需求,也即要求任务能够被快速响应、快速执行、且执行结果能够快速回传给用户,因此最小化任务完成时间是任务调度问题最主要的目标。除此之外,需要考虑的因素还涉及硬件设备功耗、分布式设备负载均衡度等。
    \item 引入了多智能终端协同技术和容器技术以后,整个系统会存在很多单一终端服务不会遇到的问题,本研究中主要解决其中的四个问题,包括基于容器化多终端服务系统架构设计、基于容器化服务资源提供技术、资源受限终端任务调度策略、基于预测的容器弹性服务策略。
\end{enumerate}
\section{本文内容安排}